% This section does not mention the possibility of applying the mechanism to clusters of different organizations
\section{Methodology}

\fbox{
    \begin{minipage}{35em}
\textbf{Thesis Statement 1:}
In a multi-cluster architecture, it is possible to increase cluster resource utilization, 
decrease average job completion time, and/or reduce average cost by incentivizing clusters to 
share resources through lending and borrowing mechanisms, 
generalizing resource lending and decoupling it from workload specificity.
    \end{minipage}
}
\fbox{
    \begin{minipage}{35em}
\textbf{Thesis Statement 2:}
In a multi-cluster architecture, it is possible to increase cluster resource utilization, 
decrease average job completion time, and/or reduce average cost by 
providing clusters with a resource trading mechanism and a pluggable policy interface allowing the 
virtual sharing of resources across clusters.
   \end{minipage}
}

% Concept introduction
% Do not communicate by sharing memory           ; instead, share memory    by communicating.
% Do not communicate by changing server ownership; instead, share resources by communicating.

% I don't want to mention capacity or whatever need because that is policy dependent
When needed, a cluster communicates with other clusters visible to it, requesting to use resources. A policy \ref{policy} 
controls when a cluster requests those resources, in addition to dictating when the other clusters accept 
providing them. In the following algorithm, the scheduler of an over-subscribed cluster tries to schedule some of 
its jobs on other clusters nodes. 

% Example Algorithm
\begin{algorithm}
\caption{Trading Scheduling Algorithm - Requester}
\begin{algorithmic}
    \For{job in WaitQueue} \Comment{sorted by X policy}
        \State $scheduled \gets ScheduleJob(job)$
        \If{$scheduled \neq True$}
        \State $BorrowResources(job)$
        \EndIf
    \EndFor

    \For{job in ReadyQueue} \Comment{sorted by X policy}
        \State $scheduled \gets ScheduleJob(job)$
        \If{$scheduled \neq True$}
        \State $WaitQueue \gets WaitQueue + job$
        \EndIf
    \EndFor
\end{algorithmic}
\end{algorithm}

\begin{algorithm}
    \caption{Trading Scheduling Algorithm - Receiver}
    \begin{algorithmic}
            \State $ resources \gets BorrowRequest $ 
            \If{$Policy(resources) = True$} \Comment{ P: if capacity $\leq 60\%$ approve}
            \State $sendAcceptRequest(job, cluster)$
            \State $prepareResources(resources)$
            \State $allowAccess$
            \EndIf
    \end{algorithmic}
    \end{algorithm} 


% Clear explanation of resource trading

% Clear explanation of pluggable policy interface
\subsection{Policy} \label{policy}
% policies that dictates decision making and incentive primitives.
Policies dictate the schedulers' trading decision-making (options?), and 
act as the knobs to tune (trading intencity?). It also includes the possible trading incentives.

% explain the 3 thesis goals (give policy examples for each goal?)

