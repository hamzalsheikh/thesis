\section{Methodology}
% Outline
The goal is to generalize borrowing and lending resources between different clusters 
without loosing the "benefits" of the multi-cluster paradigm. % assuming we're talking about clusters in same organization
This can be achieved by providing a mechanism for schedulers of different clusters to:

1- interact with each other:
When a scheduler needs to borrow resources, it has a list of possible clusters to ask
% Service mesh between clusters?  Registry
% https://www.cncf.io/blog/2021/04/12/simplifying-multi-clusters-in-kubernetes/
% https://istio.io/latest/docs/ops/deployment/deployment-models/#multiple-meshes

2- share resources with each other
Create a mechanism to allow resources to be shared between clusters.
% create a VM/container that is "managed" by the borrower for a certain amount of time

3- gain something in return
this is dependent on the policies.
% the goal is to test different policies and see which ones, coupled with the appropriate workload, 
% achieves one of the goals stated in the thesis statement.
3-1- Profit motivated. This encompass a cost function, which dictates how much a resource is "worth" for
the cluster, or if there are free resources or jobs that can be descheduled for a gain.
3-2- Cooporative (favors)
3-3- Deals