\section{Motivation and Related Work}
Data center utilization has been widely studied, with average server utilization in most data centers being low, 
ranging from 10\% to 50\%
% this paper cites other papers for the percentage of utilization 
\cite{lo_heracles_2015}. 
To combat low server utilization and diverse users, systems have been developed to allow diverse workloads to run 
on the same cluster \cite{bhattacharya_hierarchical_2013, hindman_mesos_nodate}. 
However, this causes increased interference between jobs, and thus, 
facilities are also equipped with different clusters for different purposes and teams \cite{patel_what_2022, li_lyra_2023}. 
For example, 50\% of Kubernetes \cite{verma_large-scale_2015}
end users have 10+ clusters \cite{noauthor_cncf_2023} 
, with companies like Mercedes nearing 1000 clusters \cite{noauthor_mercedes-benz_2023}.
Isolation, location, and scalability are among other reasons organizations deploy multiple clusters.
% CITE -> https://cloud.google.com/anthos/fleet-management/docs/multi-cluster-use-cases

However, the separation of resources and users into different clusters is not without a cost. 
Resource partitioning degrades performance % needs CITE
and leads to non optimal schedules because of reduced scheduler visibility.\\ % needs CITE

Researchers at ByteDance observed that their inference cluster utilization is usually around 40\% because 
of diurnal traffic patterns, and that their training cluster is always over subscribed. 
To retain the benefits of cluster separation, and reduce its cost, they introduced \textit{capacity loaning}, 
a mechanism that allows clusters to loan resources to other clusters without losing ownership.
Their policy allows the training cluster to borrow resources from the inference one, while permitting the 
inference cluster to reclaim resources dynamically. 
Compared to a baseline FIFO scheduler, capacity loaning recorded a 1.39x reduction in average queueing time and a 1.31x 
reduction in job completion time (JCT) \cite{li_lyra_2023}.
% do i still need to mention this?
% This keeps each cluster disconnected while allowing the conditional sharing of resources when needed.

% GO BEYOND THEIR WORK / EVEN MORE THAN GENERALIZATION
We plan to generalize loaning to be non-workload specific. Cluster schedulers will be able to communicate 
with each other when needed, borrowing/lending, buying/selling, or negotiating resources. % or (to trade resources)
