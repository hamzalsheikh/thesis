\section{Motivation and Related Work}
\section{Problem Statement}
Data center utilization has been widely studied, with average server utilization in most data centers being low, 
ranging from 10\% to 50\%
% this paper cites other papers for the percentage of utilization 
\cite{lo_heracles_2015}. 
To combat low server utilization and diverse users, systems have been developed to allow diverse workloads to run 
on the same cluster \cite{bhattacharya_hierarchical_2013, hindman_mesos_nodate}. 
However, this causes increased interference between jobs, and thus, 
facilities are also equipped with different clusters for different purposes and teams \cite{patel_what_2022, li_lyra_2023}. 
For example, 50\% of Kubernetes \cite{verma_large-scale_2015}
end users have 10+ clusters \cite{noauthor_cncf_2023} 
, with companies like Mercedes nearing 1000 clusters \cite{noauthor_mercedes-benz_2023}.
% Another reasons for having a multi-cluster architecture
% is business related (i.e. create teams based partitioning), and expensive provisioning and maintenance of big clusters CITE
% \section{Proposed Solution}
% We theorize that it is possible to increase resource utilization of multiple clusters through loaning. 
% The idea of resource loaning has been recently introduced, allowing over-subscribed training clusters to 
% borrow resources from traffic dependent inference clusters \cite{li_lyra_2023}.
% We plan to generalize loaning to be non-workload specific. Cluster schedulers will be able to communicate 
% with each other, borrowing/lending, buying/selling, or negotiating resources.


\subsection{Thesis Statement}
It is possible to increase cluster resource utilization, decrease average job completion time, and/or reduce average cost by incentivizing clusters
to share resources through lending and borrowing mechanisms, generalizing resource lending and decoupling it from workload specificity.