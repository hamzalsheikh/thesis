\section{Motivation and Related Work}
Data center utilization has been widely studied, with average server utilization in most data centers being low, 
ranging from 10\% to 50\%
% this paper cites other papers for the percentage of utilization 
\cite{lo_heracles_2015}. 
To combat low server utilization and diverse users, systems have been developed to allow diverse workloads to run 
on the same cluster \cite{bhattacharya_hierarchical_2013, hindman_mesos_nodate}. 
However, this causes increased interference between jobs, and thus, 
facilities are also equipped with different clusters for different purposes and teams \cite{patel_what_2022, li_lyra_2023}. 
For example, 50\% of Kubernetes \cite{verma_large-scale_2015}
end users have 10+ clusters \cite{noauthor_cncf_2023} 
, with companies like Mercedes nearing 1000 clusters \cite{noauthor_mercedes-benz_2023}.
Isolation, location, and scalability are among other reasons organizations deploy multiple clusters.
% CITE -> https://cloud.google.com/anthos/fleet-management/docs/multi-cluster-use-cases

However, the separation of resources and users into different clusters is not without a cost. 
Resource paritioning degrades performance % needs CITE
and leads to non optimal schedules because of reduced scheduler visibility.\\ % needs CITE

To retain the benefits of separating clusters, and reduce its cost, resource loaning has been recently 
introduced, allowing over-subscribed training clusters to 
borrow resources from traffic dependent inference clusters \cite{li_lyra_2023}. This keeps each cluster disconnected 
while allowing the conditional sharing of resources when needed.
We plan to generalize loaning to be non-workload specific. Cluster schedulers will be able to communicate 
with each other when needed, borrowing/lending, buying/selling, or negotiating resources. % or (to trade resources)
