\section{Thesis Plan}
% the goal is to test different policies and see which ones, coupled with the appropriate workload, 
% achieves one of the goals stated in the thesis statement.

To demonstrate the validity of the proposed mechanism, we plan to test different trading policies 
coupled with different workloads on a multi-cluster scheduling simulator \cite{sched-github}.

\subsection{Simulator}
It simulates a multi-cluster environment, where each cluster can be run locally or on different nodes. 
Clusters communicate through http, and jobs are simulated as chunks of time and resources scheduled on the clusters, 
easing the implementation of trading resources. 

\subsubsection{Components}
\begin{enumerate}
    \item Registry: The registry tracks the availability of the clusters in the environment, facilitating service discovery. 
    \item Scheduler: The main component of the simulator, it runs jobs with specific size and time, and enacts the policies 
    attached to it.
    \item Client: Sends jobs to the schedulers.
\end{enumerate}
\subsubsection{Current State}
The simulator is functional, jobs are locally scheduled in FIFO, and the first policy implemented 
is attempting to improve resource utilization and decrease job queueing time, similar to the example 
in the methodology \ref{example}. Workloads are created synthetically using probability distributions. 
Moreover, OpenTelemetry \cite{opentelemetry} and Jaeger \cite{jeager} are used to record metrics 
and traces, but still don't cover the scope of the whole project.
\\Next steps in development include:
\begin{enumerate}
    \item Develop policies' pluggable interface
    \item Production workload porting
    \item Complete telemetry and logging
\end{enumerate}
\subsection{Evaluation}
\subsubsection{Workloads}
We have two sources of workloads passed to the Client:
\begin{enumerate}
    \item Synthetic: workloads created from probability distributions
    \item Publicly available production workloads: 
        \begin{enumerate}
            \item Google Trace \cite{google-cluster-data2019}: It includes traces to 8 different clusters. 
            \item Alibaba Trace \cite{alibabaclusterdata_2023}: Includes traces of different clusters with distinct workloads.        \end{enumerate}
\end{enumerate}

\subsubsection{Policies}
We provided example policies in the methodology section, and plan to expand on them, finding policies that  
fit specific workloads to optimize the performance metrics, and will be evaluated accordingly. 

\subsubsection{Measurments}
We plan to measure the optimized-for metric per policy-workload couple. 
Metrics are both overall and per-cluster and include:
\begin{enumerate}
    \item Average JCT \ref{jct}
    \item Resource utilization \ref{example} 
    \item Cost \ref{cost} 
    \item Scheduling overhead \ref{sched-overhead}
\end{enumerate}