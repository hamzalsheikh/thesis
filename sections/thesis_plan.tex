\section{Thesis Plan}
% the goal is to test different policies and see which ones, coupled with the appropriate workload, 
% achieves one of the goals stated in the thesis statement.

To demonstrate the validity of the proposed mechanism, we plan to test different trading policies 
coupled with different workloads on a multi-cluster scheduling simulator \cite{sched-github}.

\subsection{Simulator}
Built in Go, it simulates a multi-cluster environment, where each cluster can be run locally or on different nodes. 
Clusters communicate through http, and jobs are simulated as chunks of time and resources scheduled on the clusters, 
so running jobs and sending them from one cluster to the other is simple.
\subsubsection{Components}
\begin{enumerate}
    \item Registry: The registry tracks the availability of the clusters in the environment, facilitating service discovery. 
    \item Scheduler: The main component of the simulator, it runs jobs with specific size and time, and enacts the policies 
    attached to it.
    \item Client: Sends jobs to the schedulers.
\end{enumerate}
\subsubsection{Current State}
The simulator is functional, jobs are locally scheduled in FIFO, and the first policy implemented 
is a resource utilization focused, similar to the example in the methodology \ref{example}. Moreover, 
OpenTelemetry and Jaeger are used to record metrics and traces, but still not fully functional. % CITE
Next steps include providing a pluggable interface for the policies, as well as include OpenTelemetry 
in all components to start evaluation.

\subsection{Evaluation}
\subsubsection{Workloads}
We have two sources of workloads passed to the Client:
\begin{enumerate}
    \item Synthetic: Create workloads in an attempt to find a workload-policy match
    \item Publicly available production workloads: (to be verified)
        \begin{enumerate}
            \item Google Trace: It includes traces to 8 different clusters. % https://github.com/google/cluster-data/blob/master/ClusterData2019.md
            % the alibaba 2023 version trace might work, needs further investigation
            \item Alibaba Trace: Includes traces of different clusters. % https://github.com/alibaba/clusterdata
        \end{enumerate}
\end{enumerate}

\subsubsection{Policies}
We provided example policies in the methodology section, and plan to expand on them, finding policies that might 
fit specific workloads to optimize the performance metrics. 

\subsubsection{Measurments}
Performance measurement is policy requirement specific, and thus the plan is to measure the change in the performance 
per policy. As each policy has a specific metric to optimize, we plan to measure that metric per policy-workload couple. 
Metrics would be JCT, cluster utilization, and cost. We also plan to measure the overhead of our system.