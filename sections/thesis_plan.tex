\section{Thesis Plan}
% the goal is to test different policies and see which ones, coupled with the appropriate workload, 
% achieves one of the goals stated in the thesis statement.

To demonstrate the validity of the proposed mechanism, we plan to test different trading policies 
on different workloads on a scheduling simulator \cite{sched-github}.

% if i want to implement the mechanism on kubernetes, need to mention it here
% and add an implementation section
\subsection{Simulator}
The current intuition is adding an extra layer 
of scheduling when the cluster is over-subscribed \cite{zaharia_delay_2010}, 
but the mechanism can also be implemented as an additional step in the scheduling process. 
A simulation is being built to serve as a starting point \cite{sched-github}, 
and we plan to build the project on top of Kubernetes.

I've implemented a simulation to try to mimic the how clusters can interact and test some of the policies. 
The implementation is in Go. The end goal of the simulation is the ability to deploy clusters on different 
nodes/containers and test policies.

\subsubsection{Components}
\subsubsection{Registry}
The registry tracks the availability of the schedulers, and informs clusters with it. 
This gives visibility for schedulers who need to borrow resources. 
The actual implementation should be closer to a subscription service, 
were clusters agree to participate in the market. The registry itself is not a main 
component for the actual implementation, but if we end up with a policy that needs a control service, 
or a shared resource, the registry in the simulation will play this role.

\subsubsection{Scheduler}
The scheduler is built with a cluster as a resource attached to it. 
This can also be decoupled and a scheduler can exist as a standalone node. 
The current basic implementation have the scheduler receive job requests through http requests. 
Jobs have static memory and CPU requirements and time to complete. If not enough resources 
are found on the cluster then a policy is invoked for borrowing/lending.


\subsubsection{Client}
Client sends jobs to schedulers. The client has a many-to-one relationship with schedulers
(a scheduler can have multiple clients but not visa versa).
The client is responsible for deciding the size of jobs and the rate jobs is sent to the scheduler.
Jobs are sampled from distributions:
\begin{itemize}
    \item \textbf{Job size:} Currently following a beta distribution, approximating a bounded normal distribution. 
    \item \textbf{Job send rate:} Have two options, the Weibull distribution simulating the interarrival time, and 
    Poisson distribution simulating average jobs in a time interval.
\end{itemize}
\subsection{Data Collection}
Test different policies and different synthetic workloads on the simulator. 
Run production workloads on the simulator and compare changes.

We have two sources of workloads:
\begin{enumerate}
    \item Synthetic: Create workloads in an attempt to find a workload-policy match
    \item Publicly available production workloads:
        \begin{enumerate}
            \item Google Trace % https://github.com/google/cluster-data/blob/master/ClusterData2019.md
            % the alibaba 2023 version trace might work, needs further investigation
            \item Alibaba Trace % https://github.com/alibaba/clusterdata
        \end{enumerate}
\end{enumerate}

\subsection{Evaluation}

% Variables
1- workloads % encompasses cluster util
2- policies 
3- cluster sizes

% Measurments
1- From thesis outcomes : (per cluster and total of all clusters)
    1-1- Average job completion time (AJCT)
    1-2- Average cluster utilization
    1-3- Average cost reduction
    1-4- 
2- Mechanism specific: 
   2-1- time taken to find lender
   2-2- overhead of borrowing on lender
   2-3- ... 
