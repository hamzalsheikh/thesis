\section{Thesis Plan}
% the goal is to test different policies and see which ones, coupled with the appropriate workload, 
% achieves one of the goals stated in the thesis statement.

To demonstrate the validity of the proposed mechanism, we plan to test different trading policies 
on different workloads on a scheduling simulator \cite{sched-github}.

% if i want to implement the mechanism on kubernetes, need to mention it here
% and add an implementation section

\subsection{Data Collection}
Test different policies and different synthetic workloads on the simulator. 
Run production workloads on the simulator and compare changes.

We have two sources of workloads:
\begin{enumerate}
    \item Synthetic: Create workloads in an attempt to find a workload-policy match
    \item Publicly available production workloads:
        \begin{enumerate}
            \item Google Trace % https://github.com/google/cluster-data/blob/master/ClusterData2019.md
            % the alibaba 2023 version trace might work, needs further investigation
            \item Alibaba Trace % https://github.com/alibaba/clusterdata
        \end{enumerate}
\end{enumerate}

\subsection{Policy}
% policies that dictates decision making and incentive primitives.
Policies dictate the schedulers' trading decision-making (options?), and 
act as the knobs to tune (trading intencity?). It also includes the possible trading incentives.

% a hypothetical example of a simple policy is: (if capacity is less than 80\% accepts loaning requests for 1\$ for 1 
% unit of CPU time and 1\$ per GB of memory) 
\subsection{Evaluation}
% Variables
1- workloads % encompasses cluster util
2- policies 
3- cluster sizes

% Measurments
1- From thesis outcomes : (per cluster and total of all clusters)
    1-1- Average job completion time (AJCT)
    1-2- Average cluster utilization
    1-3- Average cost reduction
    1-4- 
2- Mechanism specific: 
   2-1- time taken to find lender
   2-2- overhead of borrowing on lender
   2-3- ... 
