\section{Problem Statement}
Data center utilization has been widely studied, with average server utilization in most data centers being low, 
ranging from 10\% to 50\%
% this paper cites other papers for the percentage of utilization 
\cite{lo_heracles_2015}. 
To combat low server utilization and diverse users, systems have been developed to allow diverse workloads to run 
on the same cluster \cite{bhattacharya_hierarchical_2013, hindman_mesos_nodate}. 
However, this causes increased interference between jobs, and thus, 
facilities are also equipped with different clusters for different purposes and teams \cite{patel_what_2022, li_lyra_2023}. 
For example, 50\% of Kubernetes \cite{verma_large-scale_2015}
end users have 10+ clusters \cite{noauthor_cncf_2023} 
, with companies like Mercedes nearing 1000 clusters \cite{noauthor_mercedes-benz_2023}.
% Another reasons for having a multi-cluster architecture
% is business related (i.e. create teams based partitioning), and expensive provisioning and maintenance of big clusters CITE

\subsection{Thesis Statement}
It is possible to increase cluster resource utilization, decrease average job completion time, or reduce average cost by incentivizing clusters
to share resources through lending and borrowing mechanisms, generalizing resource lending and decoupling it from workload specificity.